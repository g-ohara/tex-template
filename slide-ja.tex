\documentclass[aspectratio=169]{slide-ja}

% For \toprule, \midrule, \bottomrule
\usepackage{booktabs}

\title{日本語スライドのテンプレート}
\author{大原玄嗣}

\begin{document}

\section{使用例}

\begin{frame}{}
  ここにフレームの内容
  \begin{itemize}
    \item アイテム1
    \item \COLabel{item2}{アイテム2}
    \item アイテム3
  \end{itemize}
  \CO{item2}{++ (2.0,-1.5)}{これはアイテム2です}
\end{frame}

\subsection{数式}

\begin{frame}{}
  \begin{align}
    \label{eq:exp}
    e^x & = 1 + x + \frac{x^2}{2!} + \frac{x^3}{3!} + \frac{x^4}{4!} + \ldots \\
        & =\sum_{k=0}^\infty \frac{x^k}{k!}
  \end{align}
  式\eqref{eq:exp}は$e^x$の\colorrect{red!20}{Taylor展開}です.
\end{frame}

\subsection{表}

\begin{frame}
  \frametitle{}
  \begin{table}
    \caption{表のキャプション}
    \begin{tabular}{ll}
      \toprule
      あい & うえお \\
      \midrule
      かき & くけこ \\
      さし & すせそ \\
      \bottomrule
    \end{tabular}
  \end{table}
\end{frame}

\appendix
\section{付録}

\begin{frame}{}
  ここに付録の内容
\end{frame}

\end{document}
