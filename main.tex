\documentclass{article}

\usepackage{listings}

\title{\texttt{tex-template}}
\author{Genji Ohara}

\begin{document}
\maketitle

\section{What is \texttt{tex-template}?}

When you write a \LaTeX{} document,
you need to compile the source code to a PDF to see how the document looks.
But the generated PDF file is not only large but also redundant
(contains information equivalent to the source code)
and should not be included in your repo.
But you may need to read it on the go or share it with someone.
Upload it to Google Drive?
No, use \verb|tex-template|.

You can easily compile your \LaTeX{} code to a PDF file locally,
and do not need to neither include it in your remote repository
nor upload it to Google Drive.
All you need to do is to tag your commit.
GitHub Actions will compile the source code to a PDF file
and add it to your release.

\section{Usage}

\subsection{Prerequisites}

\begin{itemize}
  \item \TeX{} compiler
  \item latexmk
  \item (optional) Chk\TeX{}
  \item (optional) latexindent
\end{itemize}

\subsection{How to compile locally}

\begin{enumerate}
  \item Create your repository from \verb|tex-template|
  \item Edit \verb|main.tex|
  \item Run: \verb|latexmk main.tex|
\end{enumerate}

\subsection{How to compile on GitHub}

\begin{enumerate}
  \item Tag your commit.
  \item Push the tag.
  \item Wait a few minutes.
\end{enumerate}

\section{Contributing}

Pull requests are welcome.
For major changes,
please open an issue first to discuss what you would like to change.

\section{License}

\lstinputlisting[
  basicstyle=\ttfamily\small,
  breaklines=true,
  columns=fixed,
  basewidth=0.46em,
]{LICENSE}

\end{document}

