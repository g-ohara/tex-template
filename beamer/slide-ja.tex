\documentclass{slide-ja}

% For \toprule, \midrule, \bottomrule
\usepackage{booktabs}

\title{日本語スライドのテンプレート}
\author{大原玄嗣}

\begin{document}

{%
% ignore overfull warning on title page
\vfuzz=15.63992pt
\maketitle
}

\begin{frame}
	\frametitle{フレームのタイトル}
	ここにフレームの内容
	\begin{itemize}
		\item アイテム1
		\item \COLabel{item2}{アイテム2}
		\item アイテム3
	\end{itemize}
	\CO{item2}{++ (2.0,-1.5)}{これはアイテム2です}
\end{frame}

\begin{frame}
	\frametitle{数式}
	\begin{align}
		\label{eq:exp}
		e^x & = 1 + x + \frac{x^2}{2!} + \frac{x^3}{3!} + \frac{x^4}{4!} + \ldots \\
		    & =\sum_{k=0}^\infty \frac{x^k}{k!}
	\end{align}
	式\eqref{eq:exp}は$e^x$のTaylor展開です.

\end{frame}

\begin{frame}
	\frametitle{表}
	\begin{table}
		\caption{表のキャプション}
		\begin{tabular}{ll}
			\toprule
			あい & うえお \\
			\midrule
			かき & くけこ \\
			さし & すせそ \\
			\bottomrule
		\end{tabular}
	\end{table}
\end{frame}

\end{document}
